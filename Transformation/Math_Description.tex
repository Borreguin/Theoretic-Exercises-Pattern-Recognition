\documentclass[11pt,a4paper]{article}
\usepackage[utf8]{inputenc}
\usepackage[english]{babel}
\usepackage{amsmath}
\usepackage{amsfonts}
\usepackage{amssymb}
\usepackage{graphicx}
\usepackage{soul}
\usepackage{hyperref}
\usepackage{listings}
%\lstset{language=C, breaklines=true, basicstyle=\footnotesize}
\usepackage{hyperref}

\title{Series 2a. Mean - Variance Feature - extraction  \\ Pattern Recognition }	
\author{Roberto Sanchez \\ Univesité de Fribourg}
\newtheorem{theorem}{Theorem}
\usepackage[left=2cm,right=2cm,top=2cm,bottom=2cm]{geometry}

\begin{document}

\maketitle 
\section{Description}

\subsection{A proposed feature extraction} 

The script \textbf{1.2.Transf.py} reduces the size feature vector from 28 x 28 values to 28 x 2 values. The original vector is split in 28 subsets i.e. $C_{n=[1...28]} \subseteq V^{28x28} $  where the colors values (0-255 gray scale) are not important anymore, but instead we take in count the position where each pixel has a value greater than a parameter called opacity. We define this transformation as: $X_{i} \in V^{28x28} ; \mu_{n} \in M^{28} ; \varrho_{n} \in G^{28} $, where $x_{i}$ is each feature in the vector of 28 x 28, $\mu_{n}$ is the mean position of a subset of 28 pixels and the $\varrho_{n}$ is the difference between the mean value and the variance in each subset of 28 pixels, therefore we have the equations (1) and (2) : 
   
\begin{equation}
\displaystyle  \mu_{n} = \frac{1}{N}\sum_{i=0}^{28} X_{i|v > opacity}
\end{equation} 
 

In the equation, $X_{i}$ is the gray scale feature (0-255) for each subset pixel representation, N is number of pixels that have values whose the gray scale value is greater than \textit{opacity} parameter, $\sigma$ is the variance of each subset $C_{n} \subseteq V$:  \\

\begin{equation}
\displaystyle  \sigma_{n} = \frac{1}{N}\left[ \sum_{i=0}^{28} (X_{i|v > opacity} - \mu)^2 \right] ^{1/2}
\end{equation}

Finally the transformation is the vector $Y$ defined as: $f(X_{i \in C}) \rightarrow^{T} Y_{n}=(\mu_{n},\varrho_{n})$ for each subset $C \in V$. The script returns the following structure (n: is number of 28 formed groups, therefore $n = [0... 27])$:

\begin{itemize}
	\item[x[0]]: Contains the digit handwriting data label.   	
	\item[x[2n+1]]: Contains the mean position of the pixels inside of the group n.
	\item[x[2n+2]]: Contains the difference of the mean position and the variance: $\varrho = \mu - \sigma $ 
\end{itemize}



\end{document}

