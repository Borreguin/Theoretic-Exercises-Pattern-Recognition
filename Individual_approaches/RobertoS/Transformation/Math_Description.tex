\documentclass[11pt,a4paper]{article}
\usepackage[utf8]{inputenc}
\usepackage[english]{babel}
\usepackage{amsmath}
\usepackage{amsfonts}
\usepackage{amssymb}
\usepackage{graphicx}
\usepackage{soul}
\usepackage{hyperref}
\usepackage{listings}
%\lstset{language=C, breaklines=true, basicstyle=\footnotesize}



\usepackage{listings}

\lstset{ frame=Ltb,
framerule=0pt,
aboveskip=0.5cm,
framextopmargin=3pt,
framexbottommargin=3pt,
framexleftmargin=0.4cm,
framesep=0pt,
rulesep=.4pt,
%backgroundcolor=\color{gray97},
%rulesepcolor=\color{black},
%
stringstyle=\ttfamily,
showstringspaces = false,
basicstyle=\small\ttfamily,
commentstyle=\color{gray45},
keywordstyle=\bfseries,
%
numbers=left,
numbersep=15pt,
numberstyle=\tiny,
numberfirstline = false,
breaklines=true,
}

% minimizar fragmentado de listados
\lstnewenvironment{listing}[1][]
{\lstset{#1}\pagebreak[0]}{\pagebreak[0]}

\lstdefinestyle{consola}
{basicstyle=\scriptsize\bf\ttfamily,
backgroundcolor=\color{gray75},
}

\lstdefinestyle{C}
{language=C,
}


\usepackage{hyperref}





\title{Series 2a. Mean - Variance Feature - extraction  \\ Pattern Recognition }	
\author{Roberto Sanchez \\ Univesité de Fribourg}
\newtheorem{theorem}{Theorem}
\usepackage[left=2cm,right=2cm,top=2cm,bottom=2cm]{geometry}

\begin{document}

\maketitle 
\section{Description}


\subsection{Feature Extraction} 


The script (1.2.Transf.py) reduce the vector of 28 x 28 to a smaller vector. The script \textbf{1.2.Transf.py} reduces the size vector to 28 x 2 values. The original vector is split in 28 groups where the colors values (0-255 gray scale) are not important anymore, but instead the position where each pixel has a value greater than the parameter opacity:\\

\begin{equation}
\displaystyle  \mu_{i} = \frac{1}{N}\sum_{i=0}^{28} X_{i|v > opacity}
\end{equation} 
 

N is number of pixels that have values of the gray scale greater than \textit{opacity} parameter. 

In addition, the variance is also calculate: \\

\begin{equation}
\displaystyle  \sigma_{i} = \frac{1}{N}\left[ \sum_{i=0}^{28} (X_{i|v > opacity} - \mu)^2 \right] ^{1/2}
\end{equation}

The final structure of the vector is as follow:

\begin{itemize}
	\item[x[0]]: Contains the digit handwriting data label. 
	\item[n]: Is number of 28 formed groups, therefore $n = [0... 27]$  	
	\item[x[2n+1]]: Contains the mean position of the pixels inside of the group of 28.
	\item[x[2n+2]]: Contains the rest of mean position and the variance: $\mu - \sigma $ 
\end{itemize}



\end{document}

\begin{lstlisting}[frame=single]
#include <stdio.h>
#include <sys/types.h>
#include <unistd.h>
int main ()
{
	pid_t child_pid;
	printf ("BEGIN: \nThe current process has PID: %d\n", (int) getpid ());
	child_pid = fork (); /*it creates a father and child process*/
	if (child_pid != 0) {
		printf (" This is the parent process, with ID: %d\n",(int) getpid ());
		
		printf (" This is the child's ID: %d\n",(int) child_pid);
		printf (" PING \n");
	}
	else{
		printf ("\n This is the child process, with ID: %d\n", (int) getpid ());
		printf (" pong \n");
	}
		
	return 1;
}


\end{lstlisting}
